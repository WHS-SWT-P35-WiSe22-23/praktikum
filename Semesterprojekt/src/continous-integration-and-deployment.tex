Für einen schnellen Releaseprozess sowie Arbeitsprozess in einem agilen Projekt, wie diesem, ist ein gutes CI/CD unerlässlich.
Dies ist im Folgenden unter annahme bestehender Git Repositories auf einem selbst gehosteten Gitlab Ultimate Server für die neuen Module aufgeführt.

\subsubsection{Branching Struktur}
    Unter der Annahme, dass die vorhandenen module kein Monorepo sind, werden wir diesen Aufbau beibehalten.
Somit wird es ein Repository für die App geben und eins für die Software der Hardwarekomponente.


Als Basis werden wir uns am klassischen Feature-Branching orientieren.
Es wird also in beiden Repositories Feature-- sowie Hotfix--Branches und den Develop Branch geben.
Auch soll es in beiden Fällen einen Main Branch geben, da es in beiden Fällen nur lineare Releases gibt.


Dabei wird ein Feature Branch mit upstream vom Develop erstellt, um ein neues Feature zu erstellen, einen Bug zu fixen oder sonstige Updates an Dateien zu machen.
Sind die Änderungen abgeschlossen, so wird von Develop gerebased und anschließend mit der Fast--Forward Strategie in Develop gemerged.
(Zum detaillierten Prozess mehr in Abschnitt~\ref{subsubsec:pipelines}.)
Hier wird gesquashed und Conventional Commits sind verpflichtend.
Analog verhält es sich mit Hotfix Branches, zur Main Branch.
Hotfixes dienen dabei dem schnellen fixen von kritischen Bugs in der Produktion.
Der name von Feature Branches beginnt dabei mit \qq{feature/} und der von Hotfix Branches mit \qq{hotfix/}.


Die Aufgabe vom Develop Branch ist es, die Features zusammenzuführen.
Auch findet der Hauptteil der Qualitätssicherung, der nicht in der Staging Umgebung geschieht, hier statt.
Es wird nur vor Releases in Develop commited, wenn kein neues Feature mehr gemerged werden muss.
Diese finalisieren den Release und folgen Conventional Commits.
Develop zeigt damit immer auf die aktuellste Entwicklungsversion.
Dieser wird direkt oder indirekt in Main gemerged.


Main representiert die aktuellste, öffentliche Version.
Nur von Hotfix Branches und Develop können Versionen in Main gelangen, dabei wird nicht gesquashed.
Die verschiedenen Versionen bekommen Tags mit dem Namen der entsprechenden, semantischen Version.


\paragraph{In} der App soll es zusätzlich zum Develop Branch noch einen Staging Branch geben.
    Seine Existenz wird in Abschnitt~\ref{subsubsec:deployment-umgebungen} begründet.
    Von Develop wird in Staging gemerged und von Staging anschließend in Main.


    Wichtig zu beachten ist, dass in der App Hotfixes kaum möglich sind, da die Stores auf ein manuelles Review bestehen.
    Dies dauert mindestens einen Tag, meistens sogar über zwei.
    Auch ist mit Ablehnungen zu rechnen.


\paragraph{Die} Software, der Klingel Hardware, braucht keinen Staging Branch.
    Hier wird Develop direkt in Main gemerged.

\subsubsection{Deployment-Umgebungen}\label{subsubsec:deployment-umgebungen}
    Die Entscheidung für Staging Umgebungen ist für die App ziemlich einfach.
Hier ist vieles durch den Aufbau der Stores (AppStore, PlayStore, AppGallery, etc.) vorgegeben.
Anders ist es bei Software für eigene Hardware.
Hier gild es kreative zu werden.


\paragraph{Der} Aufbau der Stores ist in vielen Punkten ähnlich:
    Ein Build wird in einem Portal hochgeladen, und kann dann verschiedenen Tracks zugewiesen werden.
    Verschiedene Gruppen von Nutzern, Testern und Entwicklern haben anschließend zugriff auf ihnen zugewiesene Tracks.
    iOS hat zwei Tracks, TestFlight for Tester und AppStore für öffentliche Releases.
    Android hingegen bietet Tracks für offene-- und geschlossene--Tests sowie öffentliche Releases.
    Ein Build kann dabei, auch im Nachhinein, mehreren Tracks zugewiesen werden.


    Unter der Annahme, dass zur laufzeit die URI des Backends angegeben werden muss, und nicht beim Kompilieren, kann auf sogenannte Flavourisierung verzichtet werden.
    Das heißt, dass \qq{Build once --- Deploy many} möglich ist.
    Entsprechend wird nur in Staging ein Build hochgeladen, da ein erneutes Hochladen in Main redundant wäre.
    Stadtessen wird der vorhandene Build einem neuen Track zugewiesen.


    Daraus folgen die Staging Umgebungen Production und Testing.
    Testing ist, je nach Projektphase ein geschlossener-- oder geöffneter--Test und bildet die Staging Branch ab.
    Production bildet hingegen Main ab und ist ein öffentlicher Release.


\paragraph{Mit} eigener Kontrolle über das Deployment wählen wir für die Software der Klingelhardware den kurzen weg zum Staging System, da voraussichtlich mit recht Produktions unähnlicher Hardware getestet werden muss;
    Develop ist die Branch, die von der Staging Umgebung abgebildet wird.
    So wird die Software über einen Update--Check, der bei jeder Anfrage an das Backend über einen Header überprüft wird, von einem dafür ausgelegten FTP Server aktualisiert.
    Diesen Server gibt es dabei für jede Environment und die URI wird in die App kompiliert.
    Die Staging Version ist auf verschiedenen Beispielsetups aufgebaut, sodass anschließend eine umfassende QA möglich ist.


    Mit demselben System wird in Main neu gebaut und die Release Version veröffentlicht.

\subsubsection{Pipelines}\label{subsubsec:pipelines}
    Mit dem von uns angenommenen Ultimate Plan von GitLab ist es möglich, sogenannte \qq{Merged results pipelines} zu erstellen.
Die Besonderheit dieser Pipelines ist, dass sie gegen einen virtuellen Commit laufen, der den Zustand der Branch darstellt, als wäre sie bereits gemerged.
So ist es möglich, das Ergebnis einer Merge Request zu analysieren, bevor wirklich gemerged wird und es entsprechend als Bedingung für einen Merge verwenden.
Aufgrund dieser Besonderheiten soll diese Form der Pipelines für unser CI genutzt werden.

\begin{description}
    \item[Mergen in Develop:]
        Beim Mergen in Develop muss im ersten Schritt Conventional Commits überprüft werden.
        Dazu dient ein entsprechendes npm Package als CLI Tool.
        Anschließend folgt die statische Codeanalyse mit einem Linter.
        Danach wird die App kompiliert und das Kompilat wird für die weitere Verwendung zwischengespeichert.
        Es folgen verschiedene Tests, gestartet mit Unit--tests, dann Golden--tests, Integration--tests und End-to-end-tests.
        Auch die Test--Coverage sollte dabei ermittelt werden.
        Wenn es zu Problemen diesbezüglich kommt, kann auch eine minimale Coverage eingestellt werden, oder als Bedingung gesetzt werden, dass diese mit jedem Merge nur steigen darf.
        Gegebenenfalls kann als Linter und für die Tests SonarCube oder eine ähnliche Software verwendet werden, um möglichst viele der Aspekte in einem Schritt abzudecken.
        Wenn all Tests und Lints erfolgreich durchgelaufen sind, dann kann die Merge Request nach einem Team-Review gemerged werden.


    \item[Mergen in Staging:]
        Wenn eine Merge Request in Staging gestartet wird, dann ist durch die Pipelines für den Develop bereits sichergestellt, dass sämtliche Lints und Tests bereits erfolgreich durchgelaufen sind.
        Somit muss hier kein besonderer Schritt mehr ausgeführt werden.


    \item[Mergen in Main:]
        Wenn es sich nicht um einen Hotfix handelt, dann müssen, genau wie bei der Merge Request in Staging, hier keine weiteren Schritte folgen.
        Handelt es sich allerdings um einen Hotfix, so müssen die Lints und Tests an dieser Stelle laufen.


    \item[Develop:]
        In den seltenen Fällen, wenn wirklich in Develop committet wird, dann sollen dieselben Schritte wie beim Mergen in Develop ausgeführt werden.
        Ist Develop die Staging Branch, so soll auch alles aus Staging ausgeführt werden.

    \item[Staging:]
        Auch in Staging müssen die Lints und Tests nicht erneut ausgeführt werden.
        Stattdessen muss auf Staging deployt werden.


        Für die App bedeutet das, dass der signierte Build in die PlayConsole (PlayStore), AppStoreConnect (AppStore), bzw. Äquivalente anderer Anbieter, hochgeladen wird.


        Die Software für die Klingel--Hardware hingegen muss auf den FTP-Server hochgeladen werden.


    \item[Main:]
        Für die Software der Klingel--Hardware muss, analog zu Staging, die Software auf den Production FTP--Server hochgeladen werden.


        Die App hingegen funktioniert nicht so analog, hier muss lediglich der Staging Build für den Release--Track der entsprechenden Platformen freigeschaltet werden.
        Die einzige Ausnahme sind Hotfixes.
        Hier muss zuvor noch eine neue Version hochgeladen werden.
\end{description}


\subsubsection{Beispiele}
    In diesem Abschnitt wollen wir den gesamten Aufbau des hier beschriebenen CI/CD Aufbaus anhand einiger praktischer Beispiele, für ein leichtes Verständnis, noch ein Mal darstellen.


    \begin{description}
        \item[Vom Feature zum Release:] Es soll ein Feature \qq{foobar} erstellt werden, sowohl in der App, als auch in der Software für die Klingel--App.
            Dieses Feature soll anschließend released werden.
            \begin{enumerate}
                \item Als erstes wird in der App die Branch \qq{feature/foobar} mit Upstream Develop erstellt.
                \item Anschließend wird das Feature implementiert und commited.
                \item Hat sich der Develop geändert, so wird die Branch auf diesem gerebased.
                \item Für die Branch wird nun in GitLab eine Merge Request zum Develop gestellt.
                \item Wenn nicht alle Tests und Lints durchlaufen, dann wird der Code korrigiert.
                \item Anschließend folgt ein Team Review durch einen Kollegen
                \item Gibt dieser sein OK, so wird die Merge Request gemerged.
                \item Sind alle anderen Features auch bereit für den Release, so wird nun eine Merge Request in Staging gestellt.
                \item Gibt das ganze (anwesende) Team sein OK, so wird die Pipeline gemerged;
                    Die App wird auf dem Testing Track veröffentlicht
                \item Da die App vorbereitet ist, kann nun auch die Entwicklung der Software für die Klingel--Hardware entwickelt werden.
                    In der Praxis kann dies auch Parallel passieren.
                    Hierzu wird auch in diesem Repository der Branch \qq{feature/foobar} mit Upstream Develop erstellt.
                \item Auch hier wird das Feature zunächst entwickelt und committed, bevor eine Merge Request in Develop gestellt wird.
                \item Anschließend folgen die Lints, Tests und das Team--Review.
                \item Da Develop auch die Funktion des Staging Branches hat, ist die Version nun in Staging System und sowohl die App, als auch die Software für die Klingel--Hardware, können ausgiebig getestet werden.
                \item Nun gibt es ein funktionierendes, getestetes System.
                    Wenn der Markt es nun hergibt, als Entwickler, wenn das Management sein Go gibt, wird in beiden Repositories eine Merge Request in Main gestellt und anschließend nach einem finalen Go der Entwicklung gemerged.
                    Damit ist die App in den Stores und auch die Hardware zeit sich die Updates vom FTP--Server.
            \end{enumerate}
        \item[Hotfix:] Unser Feature Foobar hatte leider einen Fehler in der App, der auf Faltbaren Geräten zu einem absturz der App führt.
            Dieser soll schnellstmöglich behoben werden.
            \begin{enumerate}
                \item Zunächst wird ein Branch \qq{hotfix/foobar} im App--Repository erstellt.
                \item In diesem wird der Bug nun behoben und committet.
                \item Anschließend wird eine Merge Request zurück in Main gestellt.
                \item Die Lints und Tests laufen in der Pipeline.
                \item Danach folgt das Team--Review mit Zustimmung aller (anwesender) Mitglieder.
                \item Ist die Analyse sowie das Review erfolgreich, so wird gemerged.
                    Dadurch wird die Version hochgeladen und der Review Prozess der Stores gestartet.
                \item Ist das Review der Stores erfolgreich, so wird die neue Version veröffentlicht.
            \end{enumerate}
    \end{description}
