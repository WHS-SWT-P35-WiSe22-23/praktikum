\documentclass[12pt]{article}

\usepackage[ngerman]{babel}
\usepackage{fontenc}
\usepackage{amsmath}

\begin{document}

    \section{Produktvision}\label{sec:produktvision}
    Die Kundengruppe sind nutzer der TEKLOTH Software.
    Ziel ist es, eine Videoanrufsoftware nach dem SIP Standard zu implementieren, die mit der vorhandenen Software kompatibel ist.
    In der TEKLOTH Software soll das Bild eingebunden werden und der Administrator soll den Anruf verwalten können.
    Ein angeforderter Anruf soll dabei über eine automatische Support-Kette an den höchst priorisierten Administrator, der Zeit hat, weiter geleitet werden.
    Auch soll eine Klinge funktion sowie eine Peer-to-Peer anruffunktion implementiert werden.

    \section{Persona-Stories}\label{sec:persona-stories}
    \subsection{Bewohner eines Mehrfamilienhauses}\label{subsec:bewohner-eines-mehrfamilienhauses}
    Als Bewohner möchte ich entscheiden können, ob auf meiner Klingel mein Name oder nur meine Hausnummer steht.
    Wenn die Klingel gedrückt wird möchte ich einen einstellbaren Alarm auf meinem digitalen Endgerät erhalten und das Bild vor meiner Haustür einsehen können.
    Mit dem mobilen Endgerät möchte ich auch andere Bewohner meines Wohnkomplexes erreichen können und in einer Favoritenliste sortieren können.
    Bei Notwendigkeit möchte ich auch einen Hausverwalter (Administrator) kontaktieren können.


    \subsection{Hausverwalter}\label{subsec:hausverwalter}
    Als Hausverwalter möchte das Bild der Klingel meiner Wohnungen in der TEKLOTH Software einsehen können.


    \subsection{Wartungsmitarbeiter einer Industriefirma}\label{subsec:wartungsmitarbeiter-einer-industriefirma}
    Als Wartungsmitarbeiter möchte ich in der Technik ein Gerät vorfinden, mit dem ich den Support verständigen kann.
    Der Videoanruf soll nach einer Prioritätenliste die Personen mit der besten Ahnung vom System anfragen, um den bestmöglichen Support zu erhalten.
    Dabei soll der Support Zugriff auf die entsprechende, von TEKLOTH verwaltete Technik, bekommen.


    \subsection{Hotelbesucher}\label{subsec:hotelbesucher}
    Als Hotelbesucher möchte ich den Zimmerservice kontaktieren können.
    Außerdem möchte ich mittels der Zimmernummer Endgeräte in anderen Zimmern anrufen können.


    %\bibliography{main}
    %\bibliographystyle{plain}

\end{document}
