\paragraph{Bewohner eines Mehrfamilienhauses}
    Joanna Baumann ist eine 25. Jährige Datenschutzanwältin.
    Schon immer fühlt sie sich mit ihrem Namen an der Tür unsicher.
    Und es ärgert sie, dass sie nicht sieht, wer vor der Tür steht, wenn geklingelt wird.
    Von dem neuen System erhofft sie sich ein komfortableres und sicheres Wohnen.


\paragraph{Hausverwalter}
    Zita Hoyer hat sich zu ihrem 30. Geburstag ihre erste Imobilie gekauft.
    Diese vermietet sie an Tilo.
    Das erste Jahr gab es keine Probleme, aber mit der Zeit hat sich der zugezogene Tilo Feinde gemacht.
    Jetzt wird die Haustür fast wöchentlich beschmiert.
    Zita möchte die Anlage installieren, auch um die Vandaliste zu erwischen.


\paragraph{Wartungsmitarbeiter einer Industriefirma}
    Horst Keck ist 46 Jahre alt und hat sich mit 24 Jahren, mit seiner Firma \qq{Kek Maschieneninstandhaltungen}, selbstständig gemacht.
    Mit seiner Firma wartet er unter anderem das Nuklearkraftwerk \qq{Energy Boom} in Bocholt.
    Sowohl er als auch seine Mitarbeiter sind keine Nuklearphysiker, sondern Maschinenbauer.
    Da der Betreiber des Kraftwerks aber in der Schweiz sitzt, ist es immer schwierig, die Einsätze über ein Handy zu begleiten.
    Von der Software erhofft er sich fachlichen Support von Physikern des Betreibers, wenn er an der Anlage arbeitet.


\paragraph{Hotelbesucher}
    Bernd Meier ist 27 Jahre jung und Elektriker.
    Er ist von der Elektronik GmbH auf Montage und soll in der Fachhochschule in Bocholt die Beamer neu verkabeln.
    Dazu übernachtet er im Hotel \qq{Zur guten Übernachtung}.
    Wenn er abends erschöpft von der Montage ins Hotel kommt, hat er meistens noch nicht gegessen.
    Von der Software erhofft er sich eine einfache Kommunikation mit dem Zimmerservice.
    Auch hofft er, dass er seinen Kollegen im Nebenzimmer morgen aus der Dusche klingeln kann.
