Mit dem von uns angenommenen Ultimate Plan von GitLab ist es möglich, sogenannte \qq{Merged results pipelines} zu erstellen.
Die Besonderheit dieser Pipelines ist, dass sie gegen einen virtuellen Commit laufen, der den Zustand der Branch darstellt, als wäre sie bereits gemerged.
So ist es möglich, das Ergebnis einer Merge Request zu analysieren, bevor wirklich gemerged wird und es entsprechend als Bedingung für einen Merge verwenden.
Aufgrund dieser Besonderheiten soll diese Form der Pipelines für unser CI genutzt werden.

\begin{description}
    \item[Mergen in Develop:]
        Beim Mergen in Develop muss im ersten Schritt Conventional Commits überprüft werden.
        Dazu dient ein entsprechendes npm Package als CLI Tool.
        Anschließend folgt die statische Codeanalyse mit einem Linter.
        Danach wird die App kompiliert und das Kompilat wird für die weitere Verwendung zwischengespeichert.
        Es folgen verschiedene Tests, gestartet mit Unit--tests, dann Golden--tests, Integration--tests und End-to-end-tests.
        Auch die Test--Coverage sollte dabei ermittelt werden.
        Wenn es zu Problemen diesbezüglich kommt, kann auch eine minimale Coverage eingestellt werden, oder als Bedingung gesetzt werden, dass diese mit jedem Merge nur steigen darf.
        Gegebenenfalls kann als Linter und für die Tests SonarCube oder eine ähnliche Software verwendet werden, um möglichst viele der Aspekte in einem Schritt abzudecken.
        Wenn all Tests und Lints erfolgreich durchgelaufen sind, dann kann die Merge Request nach einem Team-Review gemerged werden.


    \item[Mergen in Staging:]
        Wenn eine Merge Request in Staging gestartet wird, dann ist durch die Pipelines für den Develop bereits sichergestellt, dass sämtliche Lints und Tests bereits erfolgreich durchgelaufen sind.
        Somit muss hier kein besonderer Schritt mehr ausgeführt werden.


    \item[Mergen in Main:]
        Wenn es sich nicht um einen Hotfix handelt, dann müssen, genau wie bei der Merge Request in Staging, hier keine weiteren Schritte folgen.
        Handelt es sich allerdings um einen Hotfix, so müssen die Lints und Tests an dieser Stelle laufen.


    \item[Develop:]
        In den seltenen Fällen, wenn wirklich in Develop committet wird, dann sollen dieselben Schritte wie beim Mergen in Develop ausgeführt werden.
        Ist Develop die Staging Branch, so soll auch alles aus Staging ausgeführt werden.

    \item[Staging:]
        Auch in Staging müssen die Lints und Tests nicht erneut ausgeführt werden.
        Stattdessen muss auf Staging deployt werden.


        Für die App bedeutet das, dass der signierte Build in die PlayConsole (PlayStore), AppStoreConnect (AppStore), bzw. Äquivalente anderer Anbieter, hochgeladen wird.


        Die Software für die Klingel--Hardware hingegen muss auf den FTP-Server hochgeladen werden.


    \item[Main:]
        Für die Software der Klingel--Hardware muss, analog zu Staging, die Software auf den Production FTP--Server hochgeladen werden.


        Die App hingegen funktioniert nicht so analog, hier muss lediglich der Staging Build für den Release--Track der entsprechenden Platformen freigeschaltet werden.
        Die einzige Ausnahme sind Hotfixes.
        Hier muss zuvor noch eine neue Version hochgeladen werden.
\end{description}
