Die Entscheidung für Staging Umgebungen ist für die App ziemlich einfach.
Hier ist vieles durch den Aufbau der Stores (AppStore, PlayStore, AppGallery, etc.) vorgegeben.
Anders ist es bei Software für eigene Hardware.
Hier gild es kreative zu werden.


\paragraph{Der} Aufbau der Stores ist in vielen Punkten ähnlich:
    Ein Build wird in einem Portal hochgeladen, und kann dann verschiedenen Tracks zugewiesen werden.
    Verschiedene Gruppen von Nutzern, Testern und Entwicklern haben anschließend zugriff auf ihnen zugewiesene Tracks.
    iOS hat zwei Tracks, TestFlight for Tester und AppStore für öffentliche Releases.
    Android hingegen bietet Tracks für offene-- und geschlossene--Tests sowie öffentliche Releases.
    Ein Build kann dabei, auch im Nachhinein, mehreren Tracks zugewiesen werden.


    Unter der Annahme, dass zur laufzeit die URI des Backends angegeben werden muss, und nicht beim Kompilieren, kann auf sogenannte Flavourisierung verzichtet werden.
    Das heißt, dass \qq{Build once --- Deploy many} möglich ist.
    Entsprechend wird nur in Staging ein Build hochgeladen, da ein erneutes Hochladen in Main redundant wäre.
    Stadtessen wird der vorhandene Build einem neuen Track zugewiesen.


    Daraus folgen die Staging Umgebungen Production und Testing.
    Testing ist, je nach Projektphase ein geschlossener-- oder geöffneter--Test und bildet die Staging Branch ab.
    Production bildet hingegen Main ab und ist ein öffentlicher Release.


\paragraph{Mit} eigener Kontrolle über das Deployment wählen wir für die Software der Klingelhardware den kurzen weg zum Staging System, da voraussichtlich mit recht Produktions unähnlicher Hardware getestet werden muss;
    Develop ist die Branch, die von der Staging Umgebung abgebildet wird.
    So wird die Software über einen Update--Check, der bei jeder Anfrage an das Backend über einen Header überprüft wird, von einem dafür ausgelegten FTP Server aktualisiert.
    Diesen Server gibt es dabei für jede Environment und die URI wird in die App kompiliert.
    Die Staging Version ist auf verschiedenen Beispielsetups aufgebaut, sodass anschließend eine umfassende QA möglich ist.


    Mit demselben System wird in Main neu gebaut und die Release Version veröffentlicht.